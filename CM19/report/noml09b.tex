\documentclass[10pt,twoside,book,a5paper]{ncc}
\usepackage[utf8]{inputenc}
\usepackage[english,italian,russian]{babel}
\usepackage{indentfirst}

% This is a macro definition (version from 2016 Feb. 17) for a LaTeX template 
% for preparing documents for All-Russian Scientific Conference 
% of the Mathematical Modeling and Boundary Value Problems 
% [Matem. Mod. Kraev. Zadachi, Samara, Russian Federation]. 
%
% It was submitted by an author writing for 
% the 10th All-Russian Scientific Conference with 
% international participation (MMiKZ’16).
%
% Author: Mikhail N. Saushkin (msaushkin@gmail.com)
% License:  LaTeX Project Public License (LPPL) 
% <http://latex-project.org/lppl/>

% Dear authors of the MMiKZ’16 Conference, please do not make changes into this file.

\usepackage[a5paper, mag=1000, left=1.5cm, right=1.5cm, top=1cm, bottom=2cm, headsep=0.7cm, footskip=1cm]{geometry}
\usepackage{ifthen}
\usepackage{xcolor}
\usepackage{amsbib}
\usepackage{amsmath}
\usepackage{amssymb}
%\usepackage[colorlinks]{hyperref} 
% this package can by loaded in the amsbib package
\usepackage{nccfancyhdr}
\usepackage{mathtools}
\mathtoolsset{showonlyrefs}
\usepackage{epstopdf}
\usepackage[sort,compress]{cite}
%\usepackage{textcase} 
% this does not work with an utfx inputenc codepage for cyrillic letters

\makeatletter\@logosfalse\makeatother % we don't show amsbib package logos

\pagestyle{fancy}
\fancyhead{}
\fancyhead[RO,LE]{}
\fancyfoot{}
\fancyfoot[LE,RO]{\thepage}
\fancyfoot[LO,CE]{}
\fancyfoot[CO,RE]{}
\renewcommand{\headrulewidth}{0pt}
\renewcommand{\footrulewidth}{0pt}

\newcommand{\udc}{}
\newcommand{\theudc}{}
\renewcommand{\udc}[1]{%
\renewcommand{\theudc}{{\noindent\small \CYRU\CYRD\CYRK~#1}}}

\newcommand{\thetitle}{}
\newcommand{\TitleInTitlePage}{}
\newcommand{\TitleInTableOfContenst}{}
\renewcommand{\title}[1]{%
\renewcommand{\thetitle}{#1}%
%\renewcommand{\TitleInTitlePage}{{\small\bf\MakeTextUppercase{#1}}} 
% this does not work with an utfx inputenc codepage for cyrillic letters
\renewcommand{\TitleInTitlePage}{{\large\sc{#1}}}%
\renewcommand{\TitleInTableOfContenst}{\noindent {#1}}}

\newcommand{\theauthor}{}
\newcommand{\AutorInTitlePage}{}
\newcommand{\AuthorInTableOfContenst}{}
\renewcommand{\author}[2]{%
\renewcommand{\theauthor}{{\footnotesize\sl #1}}%
\renewcommand{\AutorInTitlePage}{{\noindent{\emph{#1}}}}%
\renewcommand{\AuthorInTableOfContenst}{#2}}

\newcommand{\ac}[2]{\addcontentsline{toc}{chapter}{{\it #1} #2}}

\newcounter{figure} % is it a ncc class trouble?
\newcounter{table}  % is it a ncc class trouble?
\newcounter{thm}
\newcounter{rem}
\newcounter{exa}
\newcounter{lem}
\newcounter{def}

\newcommand{\clean}{
\setcounter{section}{0}
\setcounter{thm}{0}
\setcounter{rem}{0}
\setcounter{exa}{0}
\setcounter{lem}{0}
\setcounter{def}{0}
\setcounter{equation}{0}
\setcounter{figure}{0}
\setcounter{table}{0}
\setcounter{footnote}{0}
}

\providecommand\phantomsection{} 

\renewcommand{\maketitle}{%
\vspace{1cm plus 1ex minus .2ex}
\phantomsection
\ac{\AuthorInTableOfContenst~\nobreak}{\TitleInTableOfContenst} 
\clean
\theudc
\begin{center}
\AutorInTitlePage 
\vskip1mm 
\TitleInTitlePage
\end{center} 
\normalsize 
\normalfont
\vskip1mm}

\newcommand{\email}[1]{\href{mailto:#1}{\texttt{\nolinkurl{#1}}}}

\DeclareSection*{1}{section}{}{0.5ex plus 1ex minus .2ex}{0.3ex plus.2ex}{\normalsize\bff}
\DeclareSection{-1}{figure}{\rm}{2.5ex}{0pt}{\small}
\DeclareSection{-2}{table}{\small\sc}{0pt}{0.5ex}{\small}
\renewcommand{\thempfootnote}{\it\alph{mpfootnote}} % 
% a \ralph command is default in the russian definition for \thempfootnote command, 
% so we have a trouble with this in the mathematical mode
\renewcommand{\theequation}{\arabic{equation}}
\renewcommand{\thesection}{\arabic{section}}
\renewcommand{\thefigure}{\rm \arabic{figure}}
\DeclareTOCEntry{0}{}{}{9.9}{}
\setcounter{tocdepth}{0}
\SectionTagSuffix{.~}
\sectionstyle{center}
\captiontagstyle[table]{right}
\captionstyle{centerlast}

\newcommand{\Proof}{\vskip1mm\textit{\CYRD\cyro\cyrk\cyra\cyrz\cyra\cyrt\cyre\cyrl\cyrsftsn\cyrs\cyrt\cyrv\cyro\/~}}
\renewenvironment{proof}[1][s]
{\vskip1mm \ifthenelse{\equal{#1}{s}}{{\it \CYRD\cyro\cyrk\cyra\cyrz\cyra\cyrt\cyre\cyrl\cyrsftsn\cyrs\cyrt\cyrv\cyro\/}.~}{{#1. }}}{\hfill$\scriptstyle\square$
\vskip1mm}

\newenvironment{newthm}[2][s]
{\vskip1mm \ifthenelse{\equal{#1}{s}}{{\small \sc #2.}}{{\small \sc #2 #1.}}}{\vskip1mm}

\renewenvironment{theorem}[1][s]
{\vskip1mm\ifthenelse{\equal{#1}{s}}{\refstepcounter{thm}{\small\sc\CYRT\cyre\cyro\cyrr\cyre\cyrm\cyra~\thethm.~}\it}
{{\small\sc\CYRT\cyre\cyro\cyrr\cyre\cyrm\cyra~#1.}\it}}{\vskip1mm}
\newenvironment{theorem*}
{\vskip1mm{\small\sc\CYRT\cyre\cyro\cyrr\cyre\cyrm\cyra.}\it}{\vskip1mm}

\renewenvironment{remark}[1][s]
{\vskip1mm \ifthenelse{\equal{#1}{s}}{\refstepcounter{rem}{\small\sc\CYRZ\cyra\cyrm\cyre\cyrch\cyra\cyrn\cyri\cyre~\therem.~}}
{{\small\sc\CYRZ\cyra\cyrm\cyre\cyrch\cyra\cyrn\cyri\cyre~#1.~}}}{\vskip1mm}
\newenvironment{remark*}
{\vskip1mm{\small\sc\CYRZ\cyra\cyrm\cyre\cyrch\cyra\cyrn\cyri\cyre.~}}{\vskip1mm}

\renewenvironment{example}[1][s]
{\vskip1mm \ifthenelse{\equal{#1}{s}}{\refstepcounter{exa}{\small\sc\CYRP\cyrr\cyri\cyrm\cyre\cyrr~\theexa.}}
{{\small\sc\CYRP\cyrr\cyri\cyrm\cyre\cyrr~#1.}}}{\vskip1mm}
\newenvironment{example*}
{\vskip1mm{\small\sc\CYRP\cyrr\cyri\cyrm\cyre\cyrr.}}
{\vskip1mm}

\renewenvironment{lemma}[1][s]
{\vskip1mm \ifthenelse{\equal{#1}{s}}{\refstepcounter{lem}{\small\sc \CYRL\cyre\cyrm\cyrm\cyra~\thelem.~}}
{{\small\sc \CYRL\cyre\cyrm\cyrm\cyra~#1.}}}{\vskip1mm}
\newenvironment{lemma*}
{\vskip1mm {\small\sc \CYRL\cyre\cyrm\cyrm\cyra.}}{\vskip1mm}

\renewenvironment{definition}[1][s]
{\vskip1mm \ifthenelse{\equal{#1}{s}}{\refstepcounter{def}{\small\sc\CYRO\cyrp\cyrr\cyre\cyrd\cyre\cyrl\cyre\cyrn\cyri\cyre~\thedef.~}}
{{\small\sc\CYRO\cyrp\cyrr\cyre\cyrd\cyre\cyrl\cyre\cyrn\cyri\cyre~#1.}}}{\vskip1mm}
\newenvironment{definition*}
{\vskip1mm{\small\sc\CYRO\cyrp\cyrr\cyre\cyrd\cyre\cyrl\cyre\cyrn\cyri\cyre.~}}{\vskip1mm}
% Убедительная просьба к авторам не редактировать файл definition.tex 
% и не вводить свои макроопределения. 
% Вы можете подключить дополнительные пакеты, 
% не изменяющие работу уже подключенных.

\begin{document}

% Вы можете удалить любые комментарии, если они мешают Вам при наборе рукописи. 
\udc{517.958:530.145:512}. 
%\udc{51:061.2/.3:004.4}
% Укажите индекс УДК, соответствующий Вашей работе.

\title{ЭПР парадокс как результат несилового взаимодействия нелокальных квантовых объектов}
%\title{Название Вашей статьи}
%% Название работы пишется строчными буквами начиная с Прописной буквы.

\author{А.~Ю.~Самарин}{Самарин~А.~Ю.}
% Автор(ы) через запятую. Не рекомендуется использование более 4-х авторов.
% Первый аргумент идёт в статью, второй - в содержание сборника трудов конференции.

\maketitle	
% Команда формирует титульую страницу рукописи и записывает необходимую
% информацию в toc-файл для создания содержания сборника трудов конференции.

\index{Самарин~А.~Ю.}	
% Перечисляются участники для поимённого списка в конце сборника трудов.
% Если авторов несколько, то каждый помещается в свою команду \index, например
%\index{Саушкин~М.~Н.}
%\index{Радченко~В.~П.}

% Рекомендуется подавать материал в структурированном виде, 
% при этом рекомендуется использовать команду \section
% Если текст рукописи строго не структурирован, то его можно 
% набирать без команд секционирования. 

\section*{Введение}

Рассмотрение квантовых частиц как совокупностей материальных полей, а не материальных точек основано не только на их представлении в квантовой теории волновыми функциями, но и на том обстоятельстве, что в противном случае постулат коллапса волновой функции квантовой механики противоречит выводам релятивистской теории о невозможности перемещения материальных объектов в пространстве со скоростью, превышающей скорость света. Подтвержденная экспериментально~\cite{sau:1} возможность мгновенного взаимодействия частиц, волновые функции которых отличны от нуля в удаленных друг от друга областях пространства~\cite{sau:2}, непосредственно в момент времени после измерения какой-либо характеристики одной их частиц, не противоречит специальной теории относительности (СТО) только в том случае, если частицы представляют собой совокупности реальных материальных полей.
Другими словами, Источником ЭПР парадокса является представление квантовых частиц как материальных точек, тогда как уравнения квантовой механики определяют их как распределенные в пространстве объекты, для которых волновая функция является единственным математическим образом в теории, который соостветствует какому-либо элементу физической реальности~\cite{sau:3}. Именно благодаря распределению в пространстве материальных носителей полей частиц и возможно их мгновенное влияния друг на друга без нарушения законов релятивистской механики.
\section{Возможность реализации ЭПР парадокса в макроскопическом масштабе} 
\label{base-section}

Многочисленные работы, связанные с попыткой устранения противоречия между квантовой механикой и СТО, при сохранении представления о квантовой частице, как о материальной точке, связаны с предположением о невозможности мгновенной передачи сигнала (информации) на расстоянии с использованием явления коллапса ~\cite{sau:4}. Дело в том, что формализм нелинейной эволюции редуцированных матриц плотности, принадлежащих к одному классу эквивалентности, примененный к стохастическим ансамблям квантовых частиц, не позволяет рассчитывать на мгновенную передачу информации на макроскопическом уровне. Это обстоятельство, однако никак не устраняет противоречия между квантовой механикой и СТО на квантовом уровне, где законы релятивистской механики также должны выполняться. Однако обе теории не противоречат друг другу, если квантовая частица представляет собой распределенный в пространстве физический объект. Но в этом случае нет никаких оснований утверждать, что мгновенная передача инфомации в принципе неосуществима на макроскопическом уровне. Реализация связи быстрее скорости света (<<faster-then-light communication>>) возможна в том случае если исходные ансамбли не эквивалентны. В работе \cite{sau:5} такие ансамбли предлагалось получить клонированием в лазерном усилителе исходных фотонов, так что квантовые состояния каждого из фотонов после усилителя в точности совпадали с состоянием исходного фотона. Однако, такое клонирование оказалось в принципе невозможным согласно законам квановой механики, что нашло свое отражение в соответствующей теореме ("no cloning-theorem"). Тем не менее, вопрос принципиальной различимости ансамблей квантовых частиц, возникающих непосредственно в результате реализации модифицированного ЭПР эксперимента (то есть измерения характеристик удаленной частицы) остается открытым.

Рассмотрим возможность возникновения таких ансамблей в результате экперименита, аналогичного тому, на основании анализа которого формулируется ЭПР парадокс. Предположим сначала, что имеется система  невзаимодействующих между собой квантовых частиц $A^{k}$ с пространственными координатами $x'^{k}$, каждая из которых взаимодейсвовала  в прошлом с частицей $B$ с координатой $x''$. Пусть $\Psi'^{1}_{t_{1}}(x'^{1}_{1})$,..., $\Psi'^{n}_{t_{1}}(x'^{n}_{1})$, $\Psi''_{t_{1}}(x''_{1})$ --- волновые функции частиц до взаимодействия. После взаимодействия система будет описываеться запутанной волновой функцией, определяемой интегральным волновым уравнением.
 Согласно~\cite{sau:6} инициация регистрирующего процесса при измерении, выражается в скачке потенциальной энергии в функционалах действия на множестве виртуальных траекторий, которое определяется как видом, производимого измерения, так и измеренным значением физической величины. Обозначим такое множество $\{\gamma_{m}\}$ и представим волновую функцию системы непосредственно после измерения в виде двух слагаемых
\begin{eqnarray*}
\Psi_{t_{2}}(x'^{1}_{2},...,x'^{n}_{2},x''_{2})\approx\\\approx\exp\frac{i}{\hbar}U_{A}\varepsilon
\int\limits_{-\infty}^{\infty}\Biggl(\int\limits_{\{\gamma''_{m}\}}\exp\biggl(\frac{i}{\hbar}\int\limits_{t_{1}}^{t_{2}}\frac{m''\dot{x''}}{2}d\tau\biggr)\prod\limits_{k=1}^{n}\Psi'^{k}_{t_{2}}(x'^{k}_{2},\gamma'')d\gamma''\Biggr)\Psi''_{t_{1}}(x''_{1})dx''_{1}+\\
+\int\limits_{-\infty}^{\infty}\Biggl(\int\limits_{\{\gamma''\neq\gamma''_{m}\}}\exp\biggl(\frac{i}{\hbar}\int\limits_{t_{1}}^{t_{2}}\frac{m''\dot{x''}}{2}d\tau\biggr)\prod\limits_{k=1}^{n}\Psi'^{k}_{t_{2}}(x'^{k}_{2},\gamma'')d\gamma''\Biggr)\Psi''_{t_{1}}(x''_{1})dx''_{1},
\end{eqnarray*}
первое из которых через чрезвычайно короткий промежуток времени редукции будет полностью определять волновую функцию в виде
\begin{equation*}
\Psi_{t_{3}}(x'^{1}_{3},...,x'^{n}_{3},x''_{3})\approx\Psi''_{t_{3}}(x''_{3})\prod\limits_{k=1}^{n}\Psi'^{k}_{t_{3}}(x'^{k}_{3}).
\end{equation*}
Таким образом, после измерения волновая функция системы представляет собой произведение волновых функций составляющих ее частиц. Кроме того, волновая функция каждой частицы $A^{k}$ зависит от вида и результата измерения характеристик удаленной частицы, что принципе позволяет по средним значениям физических величин этих частиц определить не только факт и вид удаленного измерения, но и его результат. Это означает, что в соответствием с законами квантовой механики, использование нелокальности коллапса волновой функции для мгновенной передачи информации возможно.
 
Другим возиожным способом мгновенной передачи информации является использование в процессе взаимодействия с удаленной частицей мезоскопической механической системы, находящеся в состояния кошки Шрёдингера~\cite{sau:7}.




\section*{Заключение}